\section{Usos y aplicaciones}

Ya se han ejemplificado varios escenarios en los que la tecnología miracast puede resultar muy práctica. Siendo muy usadapor ejemplo para mostrar datos y presentaciones en reuniones empresariales, ya que permite conectar el pc a un proyector compatible en pocos segundos y empezar, sin cables de por medio. Tambien es muy usado en el ambito informático, ya que gracias al mirroring se pueden mostrar demos muy facilmente en cualquier momento.

En el ambito domestico, da facilidades para compartir contenido con un gran numero de personas mediante una televisión compatible, como mostrar las fotos de las vacaciones en la tele, reproducir musica por los altavoces de una fiesta, o incluse realizar videollamadas viendolas en la tele, aunque todo esto requiere tener el dispositivo emisor siempre encendido.

Por otro lado la tecnología que usa por debajo, WiFi direct es muy usada para intercambio de archivos, se habla de que incluso podría reemplazar al bluetooht, ya que puede conectar a varios dispositivos en una lan sin necesidad de router, y gracias a su funcionamiento P2P el crecimiento que podría experimentar esta LAN ees enorme, ya que los nuevos dispositivos se conectan a otros que ya hay conectados, no al que se establecio como punto de acceso original.

Todo esto se vera con mas detalle cuando estudiemos los apartados técnicos del protocolo.
