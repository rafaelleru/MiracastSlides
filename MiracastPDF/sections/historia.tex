\section{Historia}

Miracast fue anunciado en 2013 en el congreso tecnologico de las vegas conocido como CES por la Wifi Alliance, era un protocolo revolucionario que permitia compartir contenidos multimedia inalambricamente, al igual que hasta el momento se podía hacer con un cable hdmi o VGA con la inestimable ventaja de poder prescindir de los cables, ya que todo funciona inalambricamente.

La certificación Miracast tuvo un gran calado en la industria de consumo multimedia, y en pocos meses todos los grandes de la electrónica de consumo anunciaron nuevos productos compatibles con esta tecnología, como televisiones, móviles, etc.

Aunque no sería hasta Octubre del año siguiente y los meses que lo seguirían que esta tecnología viviría su mayor auge, gracias a la competencia Apple vs Google. La primera había anunciado Airplay, un estandar similar a Miracast, y Google en su intento por no quedar atrás en esa carrera tecnológica añadió Miracast al código fuente de Android, facilitando así que todos los fabricantes de su ecosistema pudiesen implementar fácilmente esta tecnología, con lo que se produjo una gran expansión de dispositivos compatibles.
El segundo gran empujón llego con la presentación de Chromecast, un dongle HDMI que conectandose al puerto HDMI de una pantalla y compartiendo el mismo WiFi que un pc o un móvil, era capaz de hacer mirroring en la pantalla, con las grandes aplicaciones que esto presenta.
A día de hoy una gran cantidad de dispositivos y aplicaciones hacen uso de Miracast para ampliar su utilidad y seguir facilitando la vida a los usuarios.
